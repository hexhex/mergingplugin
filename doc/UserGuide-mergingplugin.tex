\documentclass[a4paper,11pt]{article}
\usepackage[ngerman,english]{babel}
\usepackage[latin1]{inputenc}
\usepackage{amssymb}
\usepackage{graphicx}
\usepackage{verbatim}
\usepackage{hyperref}
\usepackage{listings}
\usepackage{amsmath}
\usepackage{color}
\usepackage{qtree}
\usepackage{appendix}
\usepackage{float}
\usepackage{boxedminipage}
\usepackage{xspace}
\usepackage{tabularx}
\usepackage{amsthm}
\usepackage{ifthen}
\usepackage{tikz}
\usepackage{tikz-qtree}
\usepackage{verbatim}
\usepackage[ruled,vlined]{algorithm2e}
\usepackage{color}

\theoremstyle{definition}
\newtheorem{example}{Example}

\lstdefinelanguage{dlvhex}{
	keywords={not},
	otherkeywords={:-,[,],(,),{,},&,.,\,},
	keywordstyle=\textbf,
	sensitive=true,
	comment=[l]{\%},
	string=[b]",
	stringstyle=\color{gray}\textsf,
	showspaces=false,
	showstringspaces=false,
	showtabs=false
}

\newenvironment{mathprogram}[1][P]
	{
		% align at implication character
		\renewcommand{\ruleimplication}{&:-&}
%		\renewcommand{\ruleimplication}{&\leftarrow&}
		\newcommand{\ruledelimiter}{1}
		\begin{eqnarray*}
%		#1 = \{
	}
	{
%		\}
		\end{eqnarray*}
	}

\lstnewenvironment{program}[3][dlvhex]
	{
		\lstset{
			language=#1,
			basicstyle=\footnotesize,
			numberstyle=\footnotesize,
			caption={#2},
			captionpos=b,
			label={#3},
			%frame=single,
			emph={:=,(,),\{,\}},
			emphstyle=\textbf,
			mathescape=true
		}
	}
	{
	}

\lstnewenvironment{exampleprogram}[1][dlvhex]
	{
		\lstset{
			language=#1,
			basicstyle=\footnotesize,
			numberstyle=\footnotesize,
			captionpos=b,
			emph={:=,(,),\{,\}},
			emphstyle=\textbf,
			mathescape=true
		}
	}
	{
	}

\newcommand{\defneg}{not\xspace}
\newcommand{\ruleimplication}{\ensuremath{\mathit{:-}}}
%\newcommand{\ruleimplication}{\ensuremath{\leftarrow}}
\DeclareMathOperator{\OpDefNeg}{not}
\newcommand{\dlv}{\textsf{DLV}\xspace }
\newcommand{\hex}{\textsf{HEX}\xspace }
\newcommand{\dlvhex}{\textsf{dlvhex}\xspace }
\newcommand{\dotff}{\textsf{dot}\xspace }
\newcommand{\inlinecode}[1]{\textsf{#1}\xspace }
\newcommand{\true}{\emph{true}\xspace }
\newcommand{\false}{\emph{false}\xspace }
\newcommand*{\bigtimes}{\mathop{\raisebox{-.5ex}{\hbox{\huge{$\times$}}}}}
	
\begin{document}

	\title{mergingplugin - User Guide}
	\date{Febrary 28, 2010}
	\author{Christoph Redl\footnote{e0525250@mail.student.tuwien.ac.at}}
	\maketitle

	The mergingplugin consists of several external atoms for \dlvhex as well as a merging plan compiler which is intended to translate \emph{merging plans} into semantically equivalent
	\hex programs.
	
	The plugin was developed as part of the master's thesis \emph{Development of a Belief Merging Framework for \dlvhex} \cite{CR10BMF}.

	
	\section{External Atoms}
	\label{sec:ExternalAtoms}

	The merging plugin provides the following external atoms.


		\subsection{Execution of Nested Programs}
		\label{sec:ExternalAtoms:Execution}


			\subsubsection{\&hex}

				$\mathit{\&hex}$ is a unary predicate with two input parameters that is intended to execute nested \hex programs.
			
					$$\mathit{\&hex}[\mathit{Prog}, \mathit{Args}](A)$$

				An evaluation will execute the hex program in variable $\mathit{Prog}$ with the \dlvhex arguments given in $\mathit{Args}$. The result is an integer value (\emph{handle})
				that \emph{represents} the program's result symbolically. That means, the numeric value is irrelevant, but it can be used to access the result later on (similar to
				pointers in programming languages).
				
				Note that $\mathit{Prog}$ is expected to contain the program to execute directly as string literal and \emph{not} the filename of the program.
				
				\begin{example}
					\label{ex:Calling1}
					\begin{mathprogram}
\mathit{handle}(H) \ruleimplication \mathit{\&hex}[``a. b. c :- a.", ``"](H).
					\end{mathprogram}
				\end{example}

				In case the program to embed contains double quotation marks ("), they must be represented with the escape sequence $\backslash'$. The escape sequence for the backslash
				character ($\backslash$) is $\backslash\backslash$.
				
				\begin{example}
					\label{ex:Calling2}
	Assume we want to embed the program:
					\begin{mathprogram}
p(\mathit{constant}, \mathit{``string\ literal\ containing\ a\ \backslash\ backslash"}).
					\end{mathprogram}

	Then the host program looks like this:
					\begin{mathprogram}
\mathit{handle}(H) \ruleimplication \mathit{\&hex}[``p(\mathit{constant}, \\
					&&				\backslash'\mathit{string\ literal\ containing\ a\ \backslash\backslash\ backslash\backslash}').", ``"](H).
					\end{mathprogram}
				\end{example}

				
			\subsubsection{\&hexfile}

				$\mathit{\&hexfile}$ is again a unary predicate with two input parameters that is intended to execute nested \hex programs which are stored within \emph{files} in the file system.
			
					$$\mathit{\&hexfile}[\mathit{File}, \mathit{Args}](A)$$
				
				An evaluation will execture the program in the file named $\mathit{File}$ with the \dlvhex arguments given in $\mathit{Args}$. The result is a handle to the program's result.


		\subsection{Investigating the Result}
		\label{sec:ExternalAtoms:Investigation}

		
			\subsubsection{\&answersets}

				$\mathit{\&answersets}$ is a unary predicate with one input parameter.
				
					$$\mathit{\&answersets}[H](\mathit{AS})$$
				
				$H$ is expected to be a handle to a program's result (see~\ref{sec:ExternalAtoms:Execution}). Then the atom will deliver handles~$\mathit{AS}$ to each answer-set in this result.

				\begin{example}
					\label{ex:answersets}
					The program
					\begin{mathprogram}
\mathit{handle}(H, AS) \ruleimplication \mathit{\&hex}[``a. b. c :- a.", ``"](H), \\
						&&				\&answersets[H](AS).
					\end{mathprogram}
					will have one answer-set, namely~$\{ handle(0,0) \}$, where the first~$0$ is a handle to the embedded progrm's result and the second~$0$ a handle to the first answer-set
					of this program.
				\end{example}
				
				Note that answer-set handles are only unique \emph{relative} to a certain program answer. Thus, if multiple embedded programs are executed, both the handle to the program's result
				as well as the handle to an answer-set is required to uniquly identify an answer-set.

				
			\subsubsection{\&predicates}

				$\mathit{\&predicates}$ is a binary predicate with two input parameters.
				
					$$\mathit{\&predicates}[H, \mathit{AS}](\mathit{Pred}, \mathit{Arity})$$
					
				For a given handle to a program's result $H$ and a given handle to an answer-set $\mathit{AS}$, it returns tuples $(\mathit{Pred}, \mathit{Arity})$ of all predicates together with
				their arities that occur within this answer-set.

				\begin{example}
					\label{ex:predicates}
					The program
					\begin{mathprogram}
\mathit{preds}(\mathit{Pred}, \mathit{Arity}) \ruleimplication \mathit{\&hex}[``a. p(x,y).", ``"](H), \\
					&&											\&answersets[H](\mathit{AS}), \\
					&&											\&predicates[A, \mathit{AS}](\mathit{Pred}, \mathit{Arity}).
					\end{mathprogram}
					will have one answer-set, namely~$\{ preds(a,0), preds(p,2) \}$.
				\end{example}
				
				
		\subsubsection{\&arguments}

			$\mathit{\&arguments}$ is a ternary predicate with three input parameters.
			
				$$\mathit{\&arguments}[H, \mathit{AS}, \mathit{Pred}](\mathit{I}, \mathit{ArgIndex}, \mathit{Value})$$
				
			For a given predicate $\mathit{Pred}$ within a certain answer-set (identified by $H$ and $\mathit{AS}$), it will return all the information about this predicate that occurs
			within this answer-set.
			
			Each triple that is returned tells the $\mathit{Value}$ of the parameter with index $\mathit{ArgIndex}$ in the $I$-th occurrence of the predicate. $I$
			is just a running index that enables the user to distinct different occurrences of the same predicate (since a predicate can occur multiple times with different parameters). All
			triples with the same value for $I$ describe one occurrence of the predicate. The special index $s$ returns the sign of the predicate: $0$ for positive and $1$ for (strongly)
			negated.
				
			\begin{example}
				\label{ex:arguments}
				The program
				\begin{mathprogram}
\mathit{val}(\mathit{Pred}, I, \mathit{ArgIndex}, \mathit{Value}) \ruleimplication \mathit{\&hex}[``p(a,b). \neg p(x,y). q(f).", ``"](H), \\
					&&																\mathit{\&answersets}[H](\mathit{AS}), \\
					&&																\mathit{\&predicates}[A, \mathit{AS}](\mathit{Pred}, \mathit{Arity}), \\
					&&																\mathit{\&arguments}[A, \mathit{AS}, \mathit{Pred}] \\
					&&																\ \ \ \ \ (I, \mathit{ArgIndex}, \mathit{Value}).
				\end{mathprogram}
				will have one answer-set, namely
					$$\{ \mathit{val}(\textcolor{red}{p,0},\textcolor{blue}{s},\textcolor{green}{0}), \mathit{val}(\textcolor{red}{p,0},\textcolor{cyan}{0,a}), \mathit{val}(\textcolor{red}{p,0},\textcolor{magenta}{1,b}),$$
					$$\underline{\mathit{val}(p,1,s,1)}, \mathit{val}(p,1,0,x), \mathit{val}(p,1,1,y), \mathit{val}(q,0,s,0), \mathit{val}(q,0,0,f) \}$$
				
				This expresses that in the~\textcolor{red}{$0$-th occurrence of~$p$}, the~\textcolor{blue}{$s$ign} is \textcolor{green}{\emph{positive}~($0$)}, the \textcolor{cyan}{$0$-th parameter is $a$} and the \textcolor{magenta}{$1$-st parameter is $b$}.
				
				Similar for the \underline{$1$-st occurrence of $p$}, where the \underline{sign is negative}.~$q$ has just one paramter which is $f$ in the $0$-th occurrence.
			\end{example}

			
		\subsection{Operator Application}
		\label{sec:ExternalAtoms:Operators}
		
			The mergingplugin further supports the use of \emph{operators}. Operators get $n$ answers (i.e. sets of answer-sets) as input and compute a further set of answer-sets as
			output. Additionally they may get key-value pairs (over strings) as input.
			
			The predicate $\mathit{\&operator}$ is unary with three input parameters. It's output is a handle to the operator's result.
				$$\mathit{\&operator}[\mathit{OpName}, \mathit{Answers}, \mathit{KeyValuePairs}](H)$$
			$\mathit{OpName}$ is a string containing the name of the operator to apply. $\mathit{Answers}$ is a binary predicate, that contains index-handle pairs. They tell the operator
			\emph{which} answer (identified by it's handle) to pass on \emph{what} parameter position. $\mathit{KeyValuePairs}$ is a further binary predicate with key-value pairs.
			
			\begin{example}
				\label{ex:arguments}
				The program
				\begin{mathprogram}
\mathit{input}(0, H) 		\ruleimplication	\mathit{\&hex}[``a.", ``"](H). \\
\mathit{input}(1, H) 		\ruleimplication	\mathit{\&hex}[``b.", ``"](H). \\
\mathit{keyvaluepairs}(\mathit{key1}, \mathit{v1}). \\
\mathit{keyvaluepairs}(\mathit{key2}, \mathit{v2}). \\
\mathit{output}(H) 			\ruleimplication	\&operator[``\mathit{union}", input, keyvaluepairs](H). \\
\mathit{preds}(\mathit{Pred}) \ruleimplication output(H), \mathit{\&answersets}[H](AS), \\
								&&				\mathit{\&predicates}[H, AS](\mathit{Pred}, \mathit{Arity}).
				\end{mathprogram}
				executes two embedded programs, one with answer:
					$$\{ \{a\} \}$$
				and the other one with:
					$$\{ \{b\} \}$$
				Assume that operator ``union" is defined with the usual mathematical semantics. Additionally, it includes all values of the key-value pairs in the final answer.
				Then the evaluation of the $\mathit{\&operator}$ predicate will pass $\{a\}$ on the $0$-th
				parameter position and $\{b\}$ on the first one to this operator. It further passes the key-value pairs $(\mathit{key1}, \mathit{v1})$ and $(\mathit{key2}, \mathit{v2})$.
				
				The operator will compute the result $\{ a, b, \mathit{v1}, \mathit{v2} \}$, which is investigated with the $\&predicates$ evaluation. The final result of the program is
				therefore
					$$\{ \mathit{input}(0,0), \mathit{input}(1,1), \mathit{keyvaluepairs}(\mathit{key1},\mathit{v1}), \mathit{keyvaluepairs}(\mathit{key2},\mathit{v2}),$$
					$$\mathit{output}(3), \mathit{preds}(a), \mathit{preds}(b), \mathit{preds}(\mathit{v1}), \mathit{preds}(\mathit{v2}) \}$$
			\end{example}


	\section{Operator Implementation}
	\label{sec:Operators}

	
		\subsection{Operator Libraries}
	
		Operators are organized as \emph{operator libararies}, where each library can contain arbitrary many operators. An operator library must be compiled as shared object library
		that is installed either in the system or the user plugin directory of \dlvhex.
		Note: Additional plugin directories that are passed to \dlvhex using the command line argument ``--plugindir" (or ``-p") will \emph{not} be searched for operator libraries.
		However, the mergingplugin provides an own command line parameter for specifying additional operator locations (see~\ref{sec:Arguments}).
		
		Entry point of an operator library is a method with the following signature:
		\begin{center}
			\inlinecode{std::vector\textless IOperator*\textgreater OPERATORIMPORTFUNCTION()}
		\end{center}
		This method must return a vector with pointers to instances of all the operator implementations in this library (see below). the mergingplugin will call this method on
		startup and load all operators that are returned by this function.
		

		\subsection{Operator Classes}

		Operators are C++ classes (within operator libraries) that implement the interface \inlinecode{IOperator}, which is installed in the following subdirectory of the include
		directory:
		
		\begin{center}
			``dlvhex/mergingplugin/IOperator.h"
		\end{center}
	  	
		The interface defines two abstract methods, namely:
		\begin{itemize}
			\item \inlinecode{std::string getName()} \\
					The operator is expected to return it's desired name. Later, the same name is expected as parameter for the $\mathit{\&operator}$ predicate to call this operator.
					
					In case that multiple operators with the same name are defined, the mergingplugin will print a warning on startup and ignore all but the first one.
			\item \inlinecode{HexAnswer apply(int arity, std::vector\textless HexAnswer*\textgreater \& answers, OperatorArguments\& parameters) throw (OperatorException)} \\
					This method is called when the operator is actually applied. It's input is the number of answers that are passed to the operator (arity) as well as the answers
					themselves (answers). The answers are passed as vector of \inlinecode{HexAnswer}, which is defined as vector of \inlinecode{AtomSet} (since a \hex answer is a
					set of answer-sets.
					
					Finally, \inlinecode{OperatorArguments} is the set of key-value pairs. It is defined as \inlinecode{std::pair\textless string, string\textgreater}.
					
					The method is expected to return the operator's result as set of answer-sets (i.e. \inlinecode{HexAnswer}). In case of an error, an \inlinecode{OperatorException}
					can be thrown which will result in a \emph{PluginError} and thus a termination of \dlvhex.
		\end{itemize}

		
	\section{Command Line Arguments}
	\label{sec:Arguments}
	
		The mergingplugin recognizes the following command line arguments

		
		\subsection{\inlinecode{--operatorpath} or \inlinecode{--op}}
		
			Using the syntax
			\begin{center}
				\inlinecode{--operatorpath=path1,path2,...} or \inlinecode{--op=path1,path2,...}
			\end{center}
			additional paths where operators are loaded from can be specified. A path can point to a directory or a shared object library. In case of a directory, operator libs
			that are \emph{directly} within this directory will be loaded (\emph{non-recursive}!).

			
		\subsection{\inlinecode{--inputrewriter} or \inlinecode{--irw}}

			The syntax
			\begin{center}
				\inlinecode{--inputrewriter=program} or \inlinecode{--irw=program}
			\end{center}
			specifies an \emph{input rewriter}. This can be an arbitrary tool that reads from standard input and writes to standard output. The complete \dlvhex input will be
			directed through this program before reasoning starts.

			
	\section{Merging Plan Compiler}
	\label{sec:RPCompiler}
	
		The merging plan compiler is installed as part of the mergingplugin. It can be called in command line by entering:
		\begin{center}
			\inlinecode{mpcompiler}
		\end{center}
		with appropriate parameters.

		This tool translates a belief merging scenario into a \dlvhex program. The merging scenario is defined in one or more input files or is read from standard input.
		

		\subsection{Options}
		\label{sec:RPCompiler:Options}

			The command line options are:
			\begin{itemize}
				\item \inlinecode{-parsetree} \\
						Generates a parse tree rather than dlvhex code (mostly for debug tasks).
				\item \inlinecode{-help} \\
						Prints an online help message.
				\item \inlinecode{-spirit} or \inlinecode{-bison} \\
						Forces the compiler to use a \emph{boost spirit} resp. \emph{bison} generated parser. Default is spirit.
			\end{itemize}

			If no filenames are passed, the compiler will read from standard input. If at least filename is passed, standard input will \emph{not} be processed by default.
			However, if $--$ is passed as additional parameter, standard input will be read additionally to the input files.


		\subsection{Merging Plan Files}
		\label{sec:RPCompiler:Files}
		
			The merging scenario is defined in merging plan files of the following form:
			
			\begin{exampleprogram}[dlvhex]
[common signature]
predicate: pred1/arity1;
...
predicate: predN/arityN;

[belief base]
name: nameOfBeliefBase1;
mapping: "head1 :- body1."
...
mapping: "headM :- bodyM."

...

[belief base]
name: nameOfBeliefBaseK;
mapping: "head1 :- body1."
...
mapping: "headJ :- bodyJ."

[merging plan]
{
	operator: someOperatorsName;
	key1: value1;
	...
	keyN: valueN;
	source: {
		operator: subPlanOperator;
		...
		source: {nameOfBeliefBase1};
		source: {nameOfBeliefBase2};
	};
	source: {
		...
	};
}
			\end{exampleprogram}

			Essentially the file consists of 3 sections.
			
			\subsubsection{Common Signature}
			
				In statements of form
				\begin{center}
					\inlinecode{predicate: pred1/arity1;}
				\end{center}
				all relevant predicates that occur in the belief bases are defined. Those predicates will be output by \dlvhex after the merging plan was processed.

			\subsubsection{Belief Bases}
			
				Belief bases can be any data source: relational databases, XML files, etc.. The only requirement is that they are accessable from \dlvhex through an
				appropriate external atom. Belief bases are defined by blocks of form:
				\begin{exampleprogram}
[belief base]
name: nameOfBeliefBase1;
mapping: "head1 :- body1.";
...
mapping: "headM :- bodyM.";
args: "";
				\end{exampleprogram}
				where the \inlinecode{name} defines a legal name for this belief base, followed by an arbitrary number of \emph{mappings}. Mappings can essentially be arbitrary \dlvhex
				code fragments. However, in reasonable applications they access the underlying (prorietary) belief base and map their content onto the common signature (see above).
				\inlinecode{args} defines the command line arguments that shall be passed to the reasoner.
				
				Alternatively they can also be defined by
				\begin{exampleprogram}
[belief base]
name: nameOfBeliefBase1;
source: "externalfile.hex";
				\end{exampleprogram}
				where ``externalfile.hex" is an external file containing (computation source access rules and) mapping rules. Note that $\mathit{mapping}$ and $\mathit{source}$
				cannot be used simultanously.

				If an external source shall be evaluated using \dlv rather than \dlvhex, the definitions \inlinecode{dlvmapping} or \inlinecode{dlvsource} can be used, i.e.:
				\begin{exampleprogram}
[belief base]
name: nameOfDLVBeliefBase1;
mapping: "head1 :- body1.";
...
mapping: "headM :- bodyM.";
				\end{exampleprogram}
resp.
				\begin{exampleprogram}
[belief base]
name: nameOfDLVBeliefBase1;
dlvsource: "externalfile.dl";
				\end{exampleprogram}

			\subsubsection{Merging Plan}
			
				The merging plan is a hierarchical structure that combines the belief bases such that only one final result survives at the end of the day. A merging plan section is of
				form:
				\begin{exampleprogram}
operator: XYZ.
key1: value1;
...
keyN: valueN;
source: ...;
source: ...;
				\end{exampleprogram}
				Such a section defines the operator to apply, the key-value pairs that shall be passed to the operator and the sub merging plans (\inlinecode{source}). A sub merging plan
				(after a $\mathit{source}$ statement) can either be a belief base (denoted as \inlinecode{\{bbName\};}) or a \emph{composed merging} plan (i.e. the result of a prior operator application).

			\subsubsection{Syntax}
			
				The following table summarizes the complete syntax of merging task files.
						
				\begin{tabularx}{\textwidth}{p{0in}llX}
					\hline
					& \multicolumn{3}{l}{Lexer} \\
					\hline

					&	$Literal$													&	$\Rightarrow$	&	$\underline{-}? \ \Sigma_{p} \  (\underline{(}(\Sigma_{c}|\Sigma_{v}) (\underline{,} \  \Sigma_{c}|\Sigma_{v})^{*}\underline{)})?$ \\
					&	$\mathit{PredicateName}$									&	$\Rightarrow$	&	$[\underline{a}-\underline{z}] \  ([\underline{a}-\underline{z}] | [\underline{A}-\underline{Z}] | [\underline{0}-\underline{9}])^{*}$ \\
					&	$\mathit{KBName}$											&	$\Rightarrow$	&	$([\underline{a}-\underline{z}] | [\underline{A}-\underline{Z}]) \  ([\underline{a}-\underline{z}] | [\underline{A}-\underline{Z}] | [\underline{0}-\underline{9}])^{*}$ \\
					&	$\mathit{OPName}$											&	$\Rightarrow$	&	$([\underline{a}-\underline{z}] | [\underline{A}-\underline{Z}]) \  ([\underline{a}-\underline{z}] | [\underline{A}-\underline{Z}] | [\underline{0}-\underline{9}])^{*}$ \\
					&	$\mathit{Variable}$											&	$\Rightarrow$	&	$([\underline{A}-\underline{Z}]) \  ([\underline{a}-\underline{z}] | [\underline{A}-\underline{Z}] | [\underline{0}-\underline{9}])^{*}$ \\
					&	$\mathit{Number}$											&	$\Rightarrow$	&	$([\underline{1}-\underline{9}] [\underline{0}-\underline{9}]^{*}) \  | \  \underline{0}$ \\
					\hline
					& \multicolumn{3}{l}{General ASP Grammer} \\
					\hline

					&	$\mathit{Fact}$												&	$\Rightarrow$	&	$\mathit{RuleHead} \  \underline{.}$ \\
					&	$\mathit{Constraint}$										&	$\Rightarrow$	&	$\underline{:} \  \underline{-} \   \mathit{\mathit{RuleBody}} \  \underline{.}$ \\
					&	$\mathit{Query}$											&	$\Rightarrow$	&	$\underline{\mathit{not}}? \  \mathit{Literal} \   (\underline{,} \  \underline{\mathit{not}}? \  \mathit{Literal})^{*}$ \\
					&	$\mathit{RuleHead}$											&	$\Rightarrow$	&	$\mathit{Literal} \   (\underline{\vee} \   \mathit{Literal})^{*}$ \\
					&	$\mathit{RuleBody}$											&	$\Rightarrow$	&	$\mathit{Query}$ \\
					&	$\mathit{Rule}$												&	$\Rightarrow$	&	$\mathit{RuleHead} \  \underline{:} \  \underline{-} \   \mathit{RuleBody} \  \underline{.} | \mathit{Fact} | \mathit{Constraint}$ \\
					
					\hline
					& \multicolumn{3}{l}{Merging Plan Specific Grammer} \\
					\hline
					
					&	$\mathit{Program}$										&	$\Rightarrow$	&	$\mathit{CommonSigDef}$ \\
					&															&					&	$\mathit{Mappings}$ \\
					&															&					&	$\mathit{MergingPlan}$ \\
					&	$\mathit{CommonSigDef}$									&	$\Rightarrow$	&	$\underline{[\mathit{common\ signature}]}$ \\
					&															&					&	$\mathit{PredicateDefinition}^{*}$ \\
					&	$\mathit{Mappings}$										&	$\Rightarrow$	&	$\mathit{KnowledgeBase}^{*}$ \\
					&	$\mathit{MergingPlan}$									&	$\Rightarrow$	&	$\underline{[\mathit{merging\ plan}]} \  \mathit{MergingPlanNode}$ \\
					&	$\mathit{MergingPlanNode}$								&	$\Rightarrow$	&	$\underline{\{}$ \\
					&															&					&	$\  \underline{\mathit{operator}} \ \underline{:} \  \mathit{OPName} \  \underline{;}$ \\
					&															&					&	$\  (\mathit{key} \ \underline{:} \  \mathit{value} \  \underline{;})^{*}$ \\
					&															&					&	$\  (\underline{\mathit{source}} \ \underline{:} \  \mathit{MergingPlanNode} \  \underline{;})^{*}$ \\
					&															&					&	$\underline{\}} \  | \  KBName$ \\
					&	$\mathit{PredicateDefinition}$							&	$\Rightarrow$	&	$\underline{\mathit{predicate}} \ \underline{:} \  \mathit{PredicateName}  \  \underline{/} \  \mathit{Number} \  \underline{;}$ \\
					&	$\mathit{KnowledgeBase}$								&	$\Rightarrow$	&	$\underline{[\mathit{knowledge\ base}]}$ \\
					&															&					&	$\underline{\mathit{name}} \ \underline{:} \  \mathit{KBName} \  \underline{;}$ \\
					&															&					&	$(\mathit{MappingRule}^{*}) | \mathit{ExternalSource}$ \\
					&	$\mathit{MappingRule}$									&	$\Rightarrow$	&	$\underline{\mathit{mapping}} \ \underline{:} \  \underline{``} \mathit{Rule} \underline{"} \underline{;}$ \\
					&	$\mathit{ExternalSource}$								&	$\Rightarrow$	&	$\underline{\mathit{source}} \ \underline{:} \  \mathit{Filename} \underline{;}$ \\
					&	$\mathit{stringliteral}$								&	$\Rightarrow$	&	\underline{``} ${{\{"\}}^c}^{*}$ \underline{"} \\
					&															&					&	(where $S^c$ is the complement of set $S$)\\
					&	$\mathit{Filename}$										&	$\Rightarrow$	&	$\mathit{stringliteral}$ \\
					&	$\mathit{key}$											&	$\Rightarrow$	&	$\Sigma_{c} | \mathit{stringliteral}$ \\
					&	$\mathit{value}$										&	$\Rightarrow$	&	$\Sigma_{c} | \mathit{stringliteral}$ \\
					\hline
				\end{tabularx}

	\addcontentsline{toc}{chapter}{References}
	\bibliography{Bibliography}
	\bibliographystyle{plain}

\end{document}
